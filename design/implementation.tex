\documentclass[10pt,a4paper]{article}
\usepackage[a4paper]{geometry}
\usepackage{amsmath}
\usepackage{amsfonts}
\usepackage{amssymb}

% Pretty figures
\usepackage{graphicx}

\title{
    Implementation Plan \\
    BBV
}

\setlength{\parskip}{12pt}
\setlength{\parindent}{0pt}

\begin{document}

\maketitle

\begin{section}{Subsystems}
    \begin{subsection}{Authentication}
        Login with username and password or create account with Facebook. A regular
        account will be created if the user chooses to sign in with Facebook with a
        password that can later easily be changed.
        
        Regular account are linkable with Facebook id's to be able to login with them.
        They are also unlinkable.
        
        This system will use RBAC (Role Based Access Control). So each user will
        be part of a certain role group upon registration.
        Roles:
        \begin{itemize}
            \item Gast
            \item Geregistreerd
            \item Trainer
            \item Bestuur
            \item Admin
        \end{itemize}
        These role groups have a specific set of roles.
        Role groups can be created, edited and removed on runtime. They can be used to
        assign certain parts of the application to a certain group of users.
        User can only have one role group.
    \end{subsection}
    
    \begin{subsection}{Super Accounts}
        Each registered user has the option to select a user account that also has access
        to this user's account. This can be usefull for parents to be able to do stuff
        for this user.
    \end{subsection}
    
    \begin{subsection}{Pages}
        Pages are created by admins and can be assigned to certain role groups or one
        user. Pages consist of a set of widgets.
        Widgets:
        \begin{itemize}
            \item Nieuws items: These can contain a category or a set of tags from the
            news feed.
            \item Foto's: These can contain a category or a set of tags from the
            images feed.
            \item Documenten: A customizable list of documents. They can also be selected
            by category or tags.
            \item Tekst: A regular rich-text holder (with a WYSIWYG editor)-
            \item Links: Customizable set of external links.
            \item Formulier links: Buttons that refer to forms.
            \item Kalender: A filterable display of events
        \end{itemize}
        Each widget also has a customisable title.
    \end{subsection}
    
    \begin{subsection}{Item Management}
        All media items (news, images, files, forms, events) can have one category and
        multiple tags. Categories can be creation by the users, as can tags.
        When a media item is created or edited, an autocomplete is available for tags
        and categories.
        On the media item overview a list of all the categories and tags are visible.
        The overview displays a list of a fixed amount of items that are pageable and
        searchable (text input and filter by category and/or tags).
        Each item contains the CRUD (Create, Read, Update, Delete) actions.
    \end{subsection}
    
    \begin{subsection}{News}
        News items are part of a category and can have any number of tags. They also have
        one author stored. Only the author can edit the post, except for roles that have
        a higher priority. Date of creation and last edit is also included.
    \end{subsection}
    
    \begin{subsection}{Images}
        A photo album that contains categories and tags. They also have a title, author
        and date of creation. Images are searchable by their meta data.
    \end{subsection}
    
    \begin{subsection}{Files}
        Role 'Trainer' and higher has access to the files part of the application.
        They can upload any file that can be used across the application. RBAC role order
        also applies for these files. Files also can have categories (folders) and tags.
    \end{subsection}
    
    \begin{subsection}{Forms}
        Forms that can easily be created and are submitted to the database that will
        allow easy extraction of the data in csv or regular text.
    \end{subsection}
    
    \begin{subsection}{Events}
        A calendar can hold events for certain dates.
        Events can hold the following data:
        \begin{itemize}
            \item Date + Time
            \item Title
            \item Description
            \item Category
            \item Tags
            \item References:
            \begin{itemize}
                \item Documents
                \item Images
                \item Form
                \item News
            \end{itemize}
        \end{itemize}
        When references are made, the referenced item will recognise this and display
        a cross-reference to the event.
    \end{subsection}
\end{section}

\end{document}